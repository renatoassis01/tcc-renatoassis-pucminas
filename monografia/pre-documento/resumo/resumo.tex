%%%%%%%%%%%%%%%%%%%%%%%%%%%%%%%%%%%%%
%% Resumo
%% Autor: Fábio Leandro Rodrigues Cordeiro
%% Version: 1.0
%%%%%%%%%%%%%%%%%%%%%%%%%%%%%%%%%%%%%

\begin{newpage}
	\thispagestyle{empty}
	\setlength{\baselineskip}{1.5\baselineskip} % Espacamento: 1.5
	\begin{center}
		\textbf{RESUMO} \\ [1.5\baselineskip]
	\end{center}
	\singlespace
	\noindent 
	
	
	Ferramentas de gerenciamento de configuração, são basicamente framework para a construção de ambientes computacionais, permitindo a implantação e gestão de sistemas de informação em servidores. Essa construção é dada por meio de linguagens declarativas, através de instruções para a construção de cenários e suas respectivas configurações e regras.
    Este trabalho trata-se da avaliação de ferramentas de gerenciamento de configuração, destacando-se pelas características: vantagens, desvantagens, complexidade de configuração, popularidade entre os usuários, suporte, preço, documentação, plataformas suportadas e demonstração básica de funcionamento dessas.

        \vspace{1.5\baselineskip} 
	\par
        \noindent Palavras-chave: {Infraestrutura como código. Ansible, Cheff }% Palavras-chave
\end{newpage}


