%%%%%%%%%%%%%%%%%%%%%%%%%%%%%%%%%%%%%
%% Abstract
%% Autor: Fábio Leandro Rodrigues Cordeiro
%% Version: 1.0
%%%%%%%%%%%%%%%%%%%%%%%%%%%%%%%%%%%%%

{
\begin{newpage}
	\thispagestyle{empty}
	\setlength{\baselineskip}{1.5\baselineskip} % Espacamento: 1.5
	\begin{center}
		\textbf{ABSTRACT} \\ [1.5\baselineskip]
	\end{center}
	\singlespace
	\noindent 
	Configuration management tools are basically framework for the construction of computational environments, allowing the implantation and management of information systems in servers. This construction is given through declarative languages, through instructions for the construction of scenarios and their respective configurations and rules.
    This work deals with the evaluation of configuration management tools, highlighting features, advantages, disadvantages, configuration complexity, popularity among users, support, price, documentation, supported platforms and basic demonstration of their operation.
        \vspace{1.5\baselineskip} 
	\par
        \noindent Keysords: {Infra as code. Ansible. Cheff}
\end{newpage}
}
