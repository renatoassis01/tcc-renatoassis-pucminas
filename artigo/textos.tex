%%%%%%%%%%%%%%%%%%%%%%%%%%%%%%%%%%%%%%%%%%%%%%%%%%%%%%%%%%%%%%%%%%%%%%%%%%%%%%%%%%%%%%%%%%%%%%%%%%%%%%%
%%%%%%%%%%%%%% Template de Artigo Adaptado para Trabalho de Diplomação do ICEI %%%%%%%%%%%%%%%%%%%%%%%%
%% codificação UTF-8 - Abntex - Latex -  							     %%
%% Autor:    Fábio Leandro Rodrigues Cordeiro  (fabioleandro@pucminas.br)                            %% 
%% Co-autor: Prof. João Paulo Domingos Silva  e Harison da Silva                                     %%
%% Revisores normas NBR (Padrão PUC Minas): Helenice Rego Cunha e Prof. Theldo Cruz                  %%
%% Versão: 1.0     13 de março 2014                                                                  %%
%%%%%%%%%%%%%%%%%%%%%%%%%%%%%%%%%%%%%%%%%%%%%%%%%%%%%%%%%%%%%%%%%%%%%%%%%%%%%%%%%%%%%%%%%%%%%%%%%%%%%%%
\section{\esp Introdução}

Em um mercado globalizado e competitivo, as tecnologias da informação se tornaram fundamentais para o auxílio da gestão das empresas em suas atividades rotineiras e também no suporte a tomadas de decisões e a possibilidade de desenvolvimento de novos negócios, novos modelos de negócio e a modificação dos valores estratégicos das empresas. \cite{audy}.

Visando a adequação às mudanças mercadológicas, organizacionais e a manutenibilidade em relação a concorrência, o investimento em sistemas de informação tem uma grande parcela do orçamento das empresas, buscando o alinhamento da Tecnologia da informação e negócios. \cite{luftman}  
Para atender a essas necessidades, os requisitos de sistemas têm se tornado cada vez mais complexos, voláteis e robustos diante das demandas das organizações. 
As denominadas metodologias ágeis permitem a entrega de  \textit{software} funcionais em ciclos de desenvolvimento mais curtos considerando a velocidade demandada para a sua construção. \cite{sbbrocco} 
          
No entanto as equipes que lidam com a infraestrutura têm a difícil tarefa de implantar um \textit{software} em produção a medida em que são criados ou modificados, pois estes sistemas requerem dependências de componentes externos, como configurações de \textit{hardware}, sistemas operacionais, banco de dados, servidores de aplicação e web e ainda configurações específicas da aplicação. Isto demanda um tempo considerável para a implantação destes sistemas.

Para garantir um processo totalmente ágil e que acompanhe as exigências do negócio da organização é necessário que haja uma integração do desenvolvimento de sistemas e as operações de infraestrutura. Portanto essa abordagem ágil também deve ser seguida pelas equipes de infraestrutura.
O movimento cultural denominado DevOps surgido em meados de 2009, foi influenciado principalmente por metodologias ágeis e computação na nuvem. Ele teve como fundamento a automatização de processos das operações de infraestrutura e a integração entre as equipes de desenvolvimento e operações. \cite{sato}.

A automatização de processos de infraestrutura é uma das premissas do DevOps, utilizando uma abordagem para provisionar e gerenciar recursos de computação como máquinas virtuais, discos de armazenamento, regras de segurança, regras de redes e qualquer outro componente de serviço, destinando-se a simplificar significativamente a configuração e o gerenciamento de recursos em pequena e grande escala. A tradicional administração de um \textit{data center} ou uma nuvem exige que a cada alteração de \textit{software}, uma eventual ação manual na infraestrutura para acertar as configurações. Com a automação a infraestrutura também é tratada como "código" e um desenvolvedor pode descrever os recursos exigidos por uma aplicação em um arquivo padronizado de acordo com a ferramenta utilizada e os \textit{software} de gerenciamento ou provisionamento se encarregará de criar,atualizar ou remover recursos mantendo o estado da infraestrutura. 

\section{\esp Objetivos}

O objetivo deste trabalho é apresentar uma abordagem geral de \textbf{IaC}, um comparativo entre duas ferramentas de gerenciamento de configuração e apresentar os resultados obtidos e uma conclusão. 
Os objetivos específicos são explicar as diferenças entre \textit{softwares} de gerenciamento de configuração e provisionamento, infraestrutura mutável e imutável, descrever e avaliar as características das ferramentas escolhidas e apresentar resultados e conclusões.  

\section{\esp Trabalhos Relacionados}

O \textit{design} da infraestrutura é a fase do ciclo de vida do produto em que se define e configura os recursos necessários para o funcionamento dele. A Infraestrutura como código é um conjunto de práticas que usam código para configurar máquinas e redes (virtuais), instalar programas, configurar um banco de dados e definir uma regra de segurança. As práticas IaC permitem a criação de vários recursos de maneira automatizada e padronizada controlando o estado da infraestrutura, além disso permitem o controle de versão, ou seja, pode-se desfazer de qualquer mudança na infraestrutura, somente alterando o código do recursos. \cite{Carnegie}

%  \section{\esp IaC ou Infraestrutura como Código uma abordagem geral}
 
 %   \subsection{ Softwares de provisionamento }

  %  \subsection{ Softwares de gerenciamento de configuração }

   