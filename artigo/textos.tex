%%%%%%%%%%%%%%%%%%%%%%%%%%%%%%%%%%%%%%%%%%%%%%%%%%%%%%%%%%%%%%%%%%%%%%%%%%%%%%%%%%%%%%%%%%%%%%%%%%%%%%%
%%%%%%%%%%%%%% Template de Artigo Adaptado para Trabalho de Diplomação do ICEI %%%%%%%%%%%%%%%%%%%%%%%%
%% codificação UTF-8 - Abntex - Latex -  							     %%
%% Autor:    Fábio Leandro Rodrigues Cordeiro  (fabioleandro@pucminas.br)                            %% 
%% Co-autor: Prof. João Paulo Domingos Silva  e Harison da Silva                                     %%
%% Revisores normas NBR (Padrão PUC Minas): Helenice Rego Cunha e Prof. Theldo Cruz                  %%
%% Versão: 1.0     13 de março 2014                                                                  %%
%%%%%%%%%%%%%%%%%%%%%%%%%%%%%%%%%%%%%%%%%%%%%%%%%%%%%%%%%%%%%%%%%%%%%%%%%%%%%%%%%%%%%%%%%%%%%%%%%%%%%%%
\section{\esp Introdução}

Em um mercado globalizado e competitivo, as tecnologias da informação se tornaram fundamentais para o auxílio da gestão das empresas em suas atividades rotineiras e também no suporte a tomadas de decisões e a possibilidade de desenvolvimento de novos negócios, novos modelos de negócio e a modificação dos valores estratégicos das empresas. \cite{audy}.

Visando a adequação às mudanças mercadológicas, organizacionais e a manutenibilidade em relação a concorrência, o investimento em sistemas de informação tem uma grande parcela do orçamento das empresas, buscando o alinhamento da Tecnologia da informação e negócios. \cite{luftman}  
Para atender a essas necessidades, os requisitos de sistemas têm se tornado cada vez mais complexos, voláteis e robustos diante das demandas das organizações. 
As denominadas metodologias ágeis permitem a entrega de  \textit{software} funcionais em ciclos de desenvolvimento mais curtos considerando a velocidade demandada para a sua construção. \cite{sbbrocco} 
          
No entanto as equipes que lidam com a infraestrutura têm a difícil tarefa de implantar um \textit{software} em produção a medida em que são criados ou modificados, pois estes sistemas requerem dependências de componentes externos, como configurações de \textit{hardware}, sistemas operacionais, banco de dados, servidores de aplicação e web e ainda configurações específicas da aplicação. Isto demanda um tempo considerável para a implantação destes sistemas.

Para garantir um processo totalmente ágil e que acompanhe as exigências do negócio da organização é necessário que haja uma integração do desenvolvimento de sistemas e as operações de infraestrutura. Portanto essa abordagem ágil também deve ser seguida pelas equipes de infraestrutura.
O movimento cultural denominado DevOps surgido em meados de 2009, foi influenciado principalmente por metodologias ágeis e computação na nuvem. Ele teve como fundamento a automatização de processos das operações de infraestrutura e a integração entre as equipes de desenvolvimento e operações. \cite{sato}.

A automatização de processos de infraestrutura é uma das premissas do DevOps, utilizando uma abordagem para provisionar e gerenciar recursos de computação como máquinas virtuais, discos de armazenamento, regras de segurança, instalação de softeare, regras de redes e qualquer outro componente de serviço, destinando-se a simplificar significativamente a configuração e o gerenciamento de recursos em pequena e grande escala. A tradicional administração de um \textit{data center} ou uma nuvem exige que a cada alteração de \textit{software}, uma eventual ação manual na infraestrutura para acertar as configurações. Com a automação a infraestrutura também é tratada como "código" e um desenvolvedor pode descrever os recursos exigidos por uma aplicação em um arquivo padronizado de acordo com a ferramenta utilizada e os \textit{software} de gerenciamento ou provisionamento se encarregará de criar,atualizar ou remover recursos mantendo o estado da infraestrutura. 

\section{\esp Objetivos}

O objetivo deste trabalho é apresentar uma abordagem geral de \textbf{IaC}, um comparativo entre duas ferramentas de gerenciamento de configuração e apresentar os resultados obtidos e uma conclusão. 
Os objetivos específicos são explicar as diferenças entre \textit{softwares} de gerenciamento de configuração e provisionamento, infraestrutura mutável e imutável, descrever e avaliar as características das ferramentas escolhidas e apresentar resultados e conclusões. 

\section{\esp Trabalhos Relacionados}


\cite{Carnegie} explica que o \textit{design} da infraestrutura é a fase do ciclo de vida do produto em que se define e configura os recursos necessários para o funcionamento dele. A Infraestrutura como código é um conjunto de práticas que usam código para configurar recursos, como máquinas e redes (virtuais), instalar programas, configurar um banco de dados e definir uma regra de segurança. As práticas IaC permitem a criação de vários recursos de maneira automatizada e padronizada, controlando o estado da infraestrutura, além disso permitem o controle de versão, ou seja, pode-se desfazer de qualquer mudança na infraestrutura, somente alterando o código do recursos. Ele também cita que antes mesmo do surgimento da IaC os administradores de sistemas já utilizavam automação, através de \textit{script} para realizar as tarefas de configuração de infraestrutura. E que a IaC surgiu com a popularização da computação da nuvem como serviço e que todos os recursos oferecidos são virtuais. No artigo ele ainda descreve que os os provedores de serviços em nuvem fornecem um consoles de gerenciamento(uma interface \textit{web}) para o gerenciamento de recursos, porém para um sistema de larga escala, usar o console não é muito prático devido a dificuldade de gerenciar centenas de recursos que são criados e destruídos com uma grande frequência e que se deve usar as \textit{application program interface} \footnote{É um conjunto de rotinas e padrões de programação para acesso a um aplicativo de software ou plataforma baseada na Web. Permitindo que dois aplicativos se comuniquem } (API) disponibilizadas pelo provedor ou Ferramentas IaC  que interagem com essas API que resolvem a questão da criação e destruição de alta frequência desses recursos. Por fim ele cita a relação entre infraestrutura como código, \textit{Agile} e o DevOps. 

\hfill

 \citeonline{steve} fala da importância da escolha de ferramentas de IaC como (\textit{ Chef, Ansible, Puppet, SaltStack, Terraform }) se deve levar em consideração o conjunto de casos em que essas ferramentas se propõem a resolver. Estas ferramentas são classificadas em dois domínios: gerenciamento de configurações e orquestração de configuração e quais casos elas resolveram. O autor ainda explica os conceitos de infraestrutura mutável e imutável, escrita de código procedural e declarativa. Este conceitos serão abordados na seção  \textbf{IaC ou Infraestrutura como código} deste artigo.

\citeonline{steve} expressa que se deve escolher uma ferramenta focada na orquestração de infraestrutura e outra em gerenciamento de configuração de aplicativos. Parar reunir os benefícios de ambas. 

\hfill

Segundo \citeonline{masek} A popularização da virtualização e o crescente poder dos servidores e a disponibilidade da computação em nuvem levaram a um aumento significativo no número de servidores e estações de trabalho que precisam ser gerenciados. Nesse ponto, as ferramentas de gerenciamento de orquestração e gerenciamento de configuração se aplicam nesses casos. Os administradores de sistema gerenciam grupos de servidores ou estações de trabalho idênticas (hosts físicos ou máquinas virtuais) que executam aplicativos e serviços idênticos. Em seu artigo ele apresenta um estudo sobre o uso da ferramenta \textit{Ansible} nos laboratórios da \textit{Brno University of Technology (BUT)}. Ele descreve que o objetivo de utilizar uma ferramenta de IaC é fornecer uma plataforma para gerenciar com eficiência a infraestrutura de larga escala dos laboratórios universitários, com o mínimo de contribuição de desenvolvedores ou administradores.
O autor descreve a infraestrutura como código, como o gerenciamento de redes infraestrutura de trabalho (por exemplo,  em um modelo descritivo. Ele o cita dois grupo de ferramentas de infraestrutura como código mais populares \textit{Chef, Puppet, Ansible e SaltStack} são todos “ferramentas de gerenciamento de configuração”, o que significa que elas foram projetadas para instalar e gerenciar software em servidores e estações de trabalho já existentes. E as ferramentas \textit{CloudFormation, Terraform} são “ferramentas de orquestração”, o que significa que eles são projetados para provisionar os servidores.


\section{\esp IaC ou Infraestrutura como código}

Os trabalhos relacionados serviram de base para escrever esta seção. De acordo com a tabela 1 apresentada pelo autor xxxxxxx será usada para apresentar os conceitos da infraestrutura como código.




Configuration Management vs Provisioning
Mutable Infrastructure vs Immutable Infrastructure
Procedural vs Declarative
Master vs Masterless
Agent vs Agentless



%\subsection{ Softwares de provisionamento }

%\subsection{ Softwares de gerenciamento de configuração }

%\subsection{ Arquitetura }

%\subsection{ Infraestrutura }


\section{\esp Metologia}
  
 


 